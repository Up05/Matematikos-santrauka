
\subsection{Lygtys su vienu nežinomuoju}
$2x = x + 4$, $2x - x = 4$, $x = 4$ \\
Lygtyse, kai kintamasis pakeltas pirmuoju laipsnio rodikliu, kintamieji ir skaičiai turėtų būti skirtingose $=$ pusėse:
$x + 2 - 4x + 10 = 0$, $-3x = -12$, \\
Kitose lygtyse, kai kintamojo laipsnis $ > 1$, lygtį reikia prilyginti $0$.
\begin{equation}    
\begin{aligned}
    & {(x - 4)}^2 = 12 \\
    & x^2 - 8x + 16 = 12 \\
    & x^2 - 8x + 4 = 0 \dots
\end{aligned}
\qquad
\begin{aligned}
    & x(x - 4) = 0 \\
    & x = 0\arba x - 4 = 0 \\
    & x = 0\arba x = 4    
\end{aligned}
\qquad
\begin{aligned}    
    % & (x + 3)(x + 4) = 0 \\
    % & x + 3 = 0\arba x + 4 = 0 \\
    % & x = -3\arba x = -4
    & x^4 + 81 = 0 \\
    & \sqrt[4]{x^4} = \sqrt[4]{81} \\
    & x = 3 \arba  x = -3
\end{aligned}
\end{equation}

\begin{equation}
    ax^2 + bx + c = 0, \\
    D = b^2 - 4ac,     \\
    x = \frac{-b \pm \sqrt{D}}{2a}
\end{equation}


\section{Skaičių aibės}
Aibė -- nesikartojančių daiktų (dažniausia skaičių) rinkinys.
\begin{equation}
\{1; 2; -3; 3.5; \pi\}
\end{equation}

\begin{table}[h]
\begin{tabular}{c|l|l}
    
    Simb. & Reikšmė                                 & Aibės nariai/elementai \\
    \hline
    $\mathbb{N}$ & Visų natūraliųjų skaičių aibė    & $\{1; 2; 32768\:\dots\}$ \\
    $\mathbb{Z}$ & Visų sveikųjų skaičių aibė       & $\{-1; 0; 1\:\dots\}$ \\
    $\mathbb{Q}$ & Visų racionaliųjų skaičių aibė   & $\{\frac{1}{4}; -\frac{1}{16}\:\dots\}$ \\
    $\mathbb{I}$ & Visų iracionaliųjų skaičių aibė  & $\{-\pi; \sqrt{2}\:\dots\}$ \\
    $\mathbb{R}$ & Visų realiųjų skaičių aibė       & $\mathbb{Q} \cup \mathbb{I}$ \\
                 & Visų pirminių skaičių aibė       & $\{2; 3; 5; 7; 13 \dots \}$ \\
                 & Visų sudėtinių skaičių aibė      & $\{4; 6; 8; 9; 10\dots\}$ \\
    $\phi$       & Tuščia aibė                      &
\end{tabular}
\caption[Dažnai naudojamos aibės]{Dažnai naudojamos aibės}
\label{table:data}
\end{table} 

\begin{table}[h]
    \begin{tabular}{c|l|l}

        Simb. & Reikšmė                 & Pavyzdys \textit{(-iai)} \\
        \hline
        $\cup$      & Aibių sąjunga     & $\{1; 2\} \cup \{2; 3\} = \{1; 2; 3\}$ \\
        $\cap$      & Aibių sankirta    & $\{1; 2\} \cap \{2; 3\} = \{2\}$ \\
        $\setminus$ & Aibių skirtumas   & $\{1; 2\} \setminus \{2; 3\} = \{1\}$ \\
        $\subset$   & Poaibis           & $\{1; 2\} \cup \{2; 3\} = \{1; 2; 3\}$ \\
        $\in$       & Priklauso         & $2 \in \{1; 2; 3\}$, $\{1; 2\} \in \{\{1; 2\}; 3\}$ \\
    \end{tabular}
    \caption[Veiksmai su aibėmis]{Veiksmai su aibėmis}
\end{table} 

\begin{align}
    \mathbb{N} \in \mathbb{Z} \in \mathbb{Q} & \in \mathbb{R} \in \dots \\
    \mathbb{I} & \in \mathbb{R} \\
    \mathbb{Q} \cup \mathbb{I} & = \mathbb{R}
\end{align}

\subsection{Pratimai}

Kokiai mažiausiai, iš čia išvardintų, aibių \uwave{priklauso}: \\
\begin{exercises}
    \item $5$
    \item $1.3$
    \item $\{1; 2; 2\pi\}$
    \item $\left[0; 1\right]$ 
    \item $\frac{1}{x}, x \in \mathbb{N}$ 
\end{exercises} 

\subsection{Daugiau}

\href {https://en.wikipedia.org/wiki/Set-theoretic_definition_of_natural_numbers}{[EN] Natūraliųjų skaičių apibrėžimas aibių/setų teorijoje.}

