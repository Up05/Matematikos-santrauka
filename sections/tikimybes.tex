$P(A) = \frac{1}{2}$, kai $A$ -- yra koks nors įvykis ir $P(A)$ -- tikimybė, kad jis įvyks

\subsection{„Ir“ ir „arba“}

Kai naudojamas (arba norėtųsi naudoti) žodį „Ir“, tikimybes reikia dauginti vieną iš kitos:

C yra A ir B, taigi: $P(C) = P(A) * P(B)$ 
\\
Pvz.: Tris kartus metama moneta, kokia yra tikimybė, kad visus tris kartus iškris herbas?
H -- iškrito herbas. A -- iškrito herbas ir herbas ir herbas. $P(H) = \frac{1}{2}$. $P(A) = P(H)*P(H)*P(H) = \frac{1}{8}.$
