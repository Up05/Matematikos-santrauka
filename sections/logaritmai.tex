
Logaritmas duoda laipsnio rodiklį, kuriuo pagrindas turėtų būti pakeltas, kad gautusi logaritmo argumentas.
$\log_a b = c,\;a^c = b$.

\begin{math}
    \log_a b = c, jei\; a > 0, a \ne 1, b > 0
\end{math}

\begin{itemize}
    \item $a \ne 1, nes \log_1 x = c,\,x = 1^c = 1,\, c \in \mathbb{R}\; 
    (c = 1 = a, c = 2 = a, c = -3 = a \dots)$
    \item $a > 0, nes \log_{-2} x = c, x = (-2)^c,\,jei\;c = \frac{a}{b}\;ir\;b - lyginis,\;o\;a - ne{:}\; (-2)^\frac{a}{b} = \phi,\\ \hspace*{10mm} pvz{:}\; (-2)^\frac{1}{2} = \sqrt{-2}, taigi{:}\;\log_{-2} \phi = \frac{1}{2}, nors{:}\; \log_{-2} 4 = 2$
    \item $b > 0, jei\;a > 0\;arba\;b - lyginis.$
\end{itemize}

\subsection{Logaritmų savybės}
\begin{itemize}
    \item $\log_a(xy) = \log_a x + \log_a y$
    \item $\log_a(\frac{x}{y}) = \log_a x - \log_a y$
    \item $\log_a(x^k) = k\log_a x$
    \item $\log_a b = \frac{\log_c b}{\log_c a}$\qquad $(\log_a \rightarrow \log_c)$
    \item $a^{\log_a b} = b$
\end{itemize}

\subsection{Pratimai}

\begin{exercises}
    \item $\log_2(8)            $
    \item $\log_4(64)           $
    \item $\log_3\,x = 3        $
    \item $\log_2 6 + \log_2 4  $
    \item $8\log_4 2           $
    \item $\log_4(x) + 1 = 3    $
\end{exercises} 

\subsection{Daugiau}

$\log_{10} x$ arba $\lg x$ gali duoti argumento skaitmenų skaičių: $\lfloor log_{10} x \rfloor + 1$. \\
$\frac{log_e x}{log_e a} = log_a x$ duodą log su norimu pagrindu. $\frac{log_e 64}{log_e 4} = 3$.

